\documentclass{beamer}
\usepackage[utf8]{inputenc}
\usepackage[french]{babel}
\usepackage{graphics}
\usepackage{tabularx}
\usepackage[french]{babel}


%\usepackage[screen,nopanel]{pdfscreen}
%\usepackage{url}

\usetheme[progressbar=frametitle,numbering=none]{metropolis}
\definecolor{Pourpre}{RGB}{174,37,115} 
\setbeamercolor{alerted text}{fg=Pourpre}

\title{Initiation et développement de l'agilité en DUT Informatique}
\author[Thomas Clavier, Yann Secq]
{
  \{ \textbf{Thomas.CLAVIER \& Yann.SECQ} \}\texttt{@univ-lille.fr}
}
\institute{
  \includegraphics[height=13mm]{logo_iut_A_Lille.jpg}
  \includegraphics[height=13mm]{Logo_Univ_de_Lille.jpg}
}
  
\date{}

\logo{
    \includegraphics[width=1cm]{CC_BY_SA} 
}

\begin{document}

\frame{\titlepage}

\begin{frame}{Suivre nos pieds}
  \Large Si vous n’apprenez rien ou que vous ne contribuez pas, passez à autre chose ! :)
\end{frame}

\begin{frame}{Enseignement de l'agilité à l'Université}
Tribulations à l'Université d'un agiliste professionnel et de son compère enseignant-chercheur en quête d'une révolution de la pédagogie académique
   \begin{itemize}
     \item Pourquoi de l'agilité en DUT ?
     \item Comment tout a démarré
     \item Bootstrap itération par itération ...
     \item Et maintenant ?
   \end{itemize}
\end{frame}

\begin{frame}{Pourquoi de l'agilité en DUT}
   \begin{itemize}
    \item DUT $\Rightarrow$ fabrique intensive de développeurs
    \item contact direct et (assez) étroit avec les réalités professionnelles
    \item Info: algorithmique et programmation, système et réseau, base de données, méthodologie et analyse
    \item mais enseignement académique $\Rightarrow$ silos disciplinaires :(
    \item MAIS réforme du PPN  $\Rightarrow$ projet de semestre
   \end{itemize}
\end{frame}

\begin{frame}{Comment tout a démarré}
   \begin{itemize}
     \item de l'intérêt d'un réseau national structuré ...
     \item les "grandes" Journées Agiles(Toulouse'12, Aix'13, Montreuil'14)
     \item les "petites" Journées Agiles (\textbf{Lille'13}, Lyon'14)
     \item premier poste de PAST au département pour septembre 2013 !
     \item début des projets sur la lune pour Laurel et Hardy ;)
   \end{itemize}
\end{frame}

\begin{frame}{Année 1 (septembre 2013 à juillet 2014)}
  Premier projet agile sur un groupe de S3, 3 jours. 
\end{frame}

\begin{frame}{Année 2 (septembre 2014 à juillet 2015)}
  Toute la promo entre la fin du S3 et le début du S4 : 5 jours du lundi au vendredi, on vole une semaine de congés
  2 équipes sont issue de la journée entrepreneur
\end{frame}

\begin{frame}{Année 3 (septembre 2015 à juillet 2016)}
  Début de S3, S4 décallé avec 1 premier porteur de projet, puis toute la promo à la fin du S4, du mercredi au mercredi avec 6 porteurs de projets et les GEA europe, S3 décallé.
\end{frame}

\begin{frame}{Et maintenant ?}
  \begin{itemize}
    \item Laurel et Hardy, l'aventure continue ?
    \item \'Etape cruciale de la TDD en S3 ...
    \item Intégrer les GEA en début de S3 (FI/FC), puis à la journée entrepreneur, et finalement lors du projet agile de S4. 
    \item ... et peut-être un jour: DU Lean startup ? :)
  \end{itemize}
\end{frame}

\end{document}

