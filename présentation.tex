\documentclass{beamer}
\usepackage[utf8]{inputenc}
\usepackage[french]{babel}
\usepackage{graphics}
\usepackage{tabularx}
\usepackage[french]{babel}


%\usepackage[screen,nopanel]{pdfscreen}
%\usepackage{url}

\usetheme[progressbar=frametitle,numbering=none]{metropolis}
\definecolor{Pourpre}{RGB}{174,37,115} 
\setbeamercolor{alerted text}{fg=Pourpre}
%\setbeamertemplate{itemize items}{\textcolor{Pourpre}{\footnotesize$\blacksquare$}}

\title{Projet agile}
\author[Thomas Clavier, Yann Secq]
{
  Thomas Clavier \& Yann Secq 
}
\institute{
  \includegraphics[height=13mm]{logo_iut_A_Lille.jpg}
  \includegraphics[height=13mm]{Logo_Univ_de_Lille.jpg}
}
  
\date{}

\logo{
    \includegraphics[width=1cm]{cc_by_sa} 
}


\begin{document}

\frame{\titlepage}

\begin{frame}{Loi des 2 pieds : bipodocratie}
  \begin{columns}
    \begin{column}{0.5\textwidth}
      \begin{center}
        \includegraphics[width=0.6\textwidth]{foots.png}      
      \end{center}
    \end{column}
    \begin{column}{0.5\textwidth}
      Si vous n’apprenez rien ou que vous ne contribuez pas, passez à autre chose !
    \end{column}
  \end{columns}
\end{frame}

\begin{frame}{2012}
  Journée agile des IUT, Toulouse, Lille, etc. 
\end{frame}

\begin{frame}{2013 : Expérience}
  Premier projet agile, un groupe de S3, 3 jours.

  Objectifs : 
  \begin{itemize}
    \item découvrir l'agilité et l'amélioration continue à travers un vrai projet : radiateur d'information, démo, rétro, PDCA
    \item prototyper rapidement un produit web avec un serveur REST du JS, html et bootstrap.
    \item Vérifier avec un groupe que l'activité est jouable à plus grande échelle. 
  \end{itemize}
\end{frame}

\begin{frame}{2013 : Déroulement}
  \begin{itemize}
    \item 2 encadrants pour un groupe de TP (26 étudiants).
    \item légo4scrum : 2h
    \item les étudiants s'organisent entre eux pour faire des équipes de 6 ou 7.
    \item des sujets apportés par les étudiants
    \item sprint de 2h avec démo et rétro
    \item soutenance en amphie avec démo
    \item rétro de groupe sur l'activité
    \item des contraintes technique forte
  \end{itemize}
\end{frame}

\begin{frame}{2013 : Apprentissages}
  \begin{itemize}
    \item 1 enseignant pour 13 étudiants c'est très bien,
    \item la généralisation à toute la promo est envisageable
    \item avec le budget que nous avons ce sera 5 enseignants pour 4 groupes de TP.
    \item plonger les étudiants dans un contexte technique innovant avec une bonne base théorique (ie. en fin d'IUT) c'est déstabilisant et très formateur. 
    \item activité trop courte pour à la foi faire le backlog, le découper et le coder.
  \end{itemize}
\end{frame}

\begin{frame}{2014 : Éxpérience}
  Fin de S4 décalé, Antoine, 1 porteur de projet externe.

  Toute la promo entre la fin du S3 et le début du S4 : 5 jours du lundi au vendredi, on vole une semaine de congés
  2 équipes sont issue de la journée entrepreneur, 5 enseignants dont 3 non agiliste.
\end{frame}
\begin{frame}{2015 : Éxperiences}
  Début de S3, S4 décallé avec 1 premier porteur de projet, puis toute la promo à la fin du S4, du mercredi au mercredi avec 6 porteurs de projets et les GEA europe, S3 décallé.
\end{frame}
\begin{frame}{Next}
  Intégrer les GEA en début de S3 (FI/FC), les retrouver à la journée entrepreneur, et les retrouver en fin de S4. 
\end{frame}

\end{document}

