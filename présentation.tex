\documentclass{beamer}
\usepackage[utf8]{inputenc}
\usepackage[french]{babel}
\usepackage{graphics}
\usepackage{tabularx}
\usepackage[french]{babel}


%\usepackage[screen,nopanel]{pdfscreen}
%\usepackage{url}

\usetheme[progressbar=frametitle,numbering=none]{metropolis}
\definecolor{Pourpre}{RGB}{174,37,115} 
\setbeamercolor{alerted text}{fg=Pourpre}

\title{Projet agile}
\author[Thomas Clavier, Yann Secq]
{
  Thomas Clavier \& Yann Secq 
}
\institute{
  \includegraphics[height=13mm]{logo_iut_A_Lille.jpg}
  \includegraphics[height=13mm]{Logo_Univ_de_Lille.jpg}
}
  
\date{}

\logo{
    \includegraphics[width=1cm]{cc_by_sa} 
}


\begin{document}

\frame{\titlepage}

\begin{frame}{Suivre nos pieds}
  \Large Si vous n’apprenez rien ou que vous ne contribuez pas, passez à autre chose !
\end{frame}

\begin{frame}{Année -1}
  
\end{frame}
\begin{frame}{Année 0}
  Premier projet agile sur un groupe de S3, 3 jours. 
\end{frame}
\begin{frame}{Année 1}
  Toute la promo entre la fin du S3 et le début du S4 : 5 jours du lundi au vendredi, on vole une semaine de congés
  2 équipes sont issue de la journée entrepreneur
\end{frame}
\begin{frame}{Année 2}
  Début de S3, S4 décallé avec 1 premier porteur de projet, puis toute la promo à la fin du S4, du mercredi au mercredi avec 6 porteurs de projets et les GEA europe, S3 décallé.
\end{frame}
\begin{frame}{Next}
  Intégrer les GEA en début de S3 (FI/FC), les retrouver à la journée entrepreneur, et les retrouver en fin de S4. 
\end{frame}

\end{document}

